%% LyX 2.3.4.4 created this file.  For more info, see http://www.lyx.org/.
%% Do not edit unless you really know what you are doing.
\documentclass[12pt,english,format=acmsmall]{article}
\usepackage{newcent}
\usepackage[scaled=0.85]{beramono}
\usepackage[T1]{fontenc}
\usepackage[latin9]{inputenc}
\usepackage[a4paper]{geometry}
\geometry{verbose,tmargin=2.5cm,bmargin=2.5cm,lmargin=2.5cm,rmargin=2.5cm,headheight=2.5cm,headsep=2.5cm,footskip=2.5cm}
\usepackage{babel}
\usepackage{array}
\usepackage{amstext}
\usepackage{xargs}[2008/03/08]
\usepackage[unicode=true,pdfusetitle,
 bookmarks=true,bookmarksnumbered=false,bookmarksopen=false,
 breaklinks=false,pdfborder={0 0 1},backref=false,colorlinks=false]
 {hyperref}

\makeatletter

%%%%%%%%%%%%%%%%%%%%%%%%%%%%%% LyX specific LaTeX commands.
\newcommand*\LyXZeroWidthSpace{\hspace{0pt}}
%% Because html converters don't know tabularnewline
\providecommand{\tabularnewline}{\\}

%%%%%%%%%%%%%%%%%%%%%%%%%%%%%% User specified LaTeX commands.
% recommended but missing
%\usepackage{orcidlink}
%\usepackage{thumbpdf}


 

\makeatother

\usepackage{listings}
\lstset{basicstyle={\ttfamily\small},
breaklines=false,
columns=flexible,
keepspaces=true,
moredelim={**[is][\bfseries]{~}{~}}}
\renewcommand{\lstlistingname}{Listing}

\begin{document}
\title{User's Guide for UltimateKalman:\\
a Library for Flexible Kalman Filtering and Smoothing Using Orthogonal
Transformations}
\author{Sivan Toledo}

\maketitle
\global\long\def\linearestimator{F}%

\global\long\def\modelfun{M}%

\global\long\def\penaltyfun{\phi}%

\global\long\def\objectivefun{\phi}%

\global\long\def\grad{\nabla}%

\newcommandx\hessian[1][usedefault, addprefix=\global, 1=]{\nabla_{#1}^{2}}%

\global\long\def\jacobian{\mathrm{J}}%

\global\long\def\exact#1{#1}%

\global\long\def\estimate#1{\hat{#1}}%

\global\long\def\controlpoint{\rho}%

\global\long\def\location{\ell}%

\global\long\def\noise{\epsilon}%

\global\long\def\observations{b}%

\global\long\def\expectation{\operatorname{E}}%

\global\long\def\iu{\mathbf{i}}%

\global\long\def\vecone{\mathbf{1}}%

\global\long\def\veczero{\mathbf{0}}%

\global\long\def\covold{\text{cov}}%

\global\long\def\nonjsscov{\operatorname{cov}}%

\global\long\def\cov{\operatorname{cov}}%

\global\long\def\var{\text{var}}%

\global\long\def\fim{\mathcal{I}}%

\global\long\def\loglikelihood{\mathcal{L}}%

\global\long\def\score{\mathcal{S}}%

\global\long\def\duration{\vartheta}%

\global\long\def\attenuation{a}%
\global\long\def\cmplxatt{\alpha}%

\global\long\def\initialphase{\varphi}%

\global\long\def\satclockerr{\eta}%

\global\long\def\ionodelay{\psi}%

\global\long\def\tropodelay{\xi}%

\global\long\def\xcorr{\operatorname{xcorr}}%

\global\long\def\diag{\operatorname{diag}}%

\global\long\def\rank{\operatorname{rank}}%

\global\long\def\erf{\operatorname{erf}}%

\global\long\def\erfc{\operatorname{erfc}}%

\global\long\def\range{\operatorname{range}}%

\global\long\def\trace{\operatorname{trace}}%

\global\long\def\ops{\operatorname{ops}}%

\global\long\def\prob{\operatorname{Prob}}%

\global\long\def\real{\text{Re}}%

\global\long\def\imag{\text{Im}}%

\global\long\def\square{\text{\ensuremath{\blacksquare}}}%

\global\long\def\irange{\boldsymbol{:}}%

UltimateKalman is a library that implements efficient Kalman filtering
and smoothing algorithms using orthogonal transformations. The algorithms
are based on an algorithm by Paige and Saunders~\cite{PaigeSaunders:1977:Kalman},
which to the best of our knowledge, has not been implemented before.
This guide is part of an Algorithm-type article in the journal ACM
Transactions on Mathematical Software. The algorithms are described
in that article. This user guide only explains how to use the software
library.

UltimateKalman is currently available in MATLAB, C, and Java.
Each implementation is separate and does not rely on the others. The
implementation includes MATLAB adapter classes that allow invocation
of the C and Java implementations from MATLAB. This allows a single
set of test functions to test all three implementations. 

The programming interfaces of all three implementations are similar.
They offer exactly the same functionality using the same abstractions,
and each employs good programming practices of the respected language.
For example, the MATLAB and Java implementations use overloading
(using the same method name more than once, with different argument
lists). Another example is a method that returns two values in the
MATLAB implementation, but only one in the others; the second value
is returned by a separate method or function in the Java and C implementations.
The only differences are ones that are unavoidable due to the constraints
of each programming language.

The MATLAB implementation does not rely on any MATLAB toolbox, only
on functionality that is part of the core product. The implementation
also works under GNU Octave. The C implementation relies on basic
matrix and vector operations from the BLAS~\cite{BLAS,BLAS3ALG}
and on the QR and Cholesky factorizations from LAPACK~\cite{LAPACK-UG}.
The Java implementation uses the Apache Commons Math library for
both basic matrix-vector operations and for the QR and Cholesky factorizations.
The Cholesky factorization is used only to factor covariance matrices
that are specified explicitly, as opposed to being specified by inverse
factors or triangular factors.

We first describe how the different implementations represent matrices,
vectors, and covariance matrices. Then we describe in detail the MATLAB
programming interface and implementation and then comment on the differences
between them and those of the other two implementations. The guide
ends with a discussion of the data structures that are used to represent
the step sequence and a presentation of a mechanism for measuring
the performance of the implementations.

\section{\label{sec:matrices-and-vectors}The representation of vectors and
matrices}

The MATLAB implementation uses native MATLAB matrices and vectors.
The Java implementation uses the types \texttt{RealMatrix} and \texttt{RealVector}
from the Apache Commons Math library~(both are interface types with
multiple implementations). 

The C implementation defines a type called \texttt{matrix\_t} to
represent matrices and vectors. The implementation defines functions
that implement basic operations of matrices and vectors of this type.
The type is implemented using a structure that contains a pointer
to an array of double-precision elements, which are stored columnwise
as in the BLAS and LAPACK, and integers that describe the number
of rows and columns in the matrix and the stride along rows (the so-called
leading dimension in the BLAS and LAPACK interfaces). To avoid name-space
pollution, in client code this type is called \texttt{kalman\_matrix\_t}.

\section{\label{sec:covariance-matrices}The representation of covariance
matrices}

Like all Kalman filters, UltimateKalman consumes covariance matrices
that describe the distribution of the error terms and produces covariance
matrices that describe the uncertainty in the state estimates $\estimate u_{i}$.
The input covariance matrices are not used explicitly; instead, the
inverse factor $W$ of a covariance matrix $C=(W^{T}W)^{-1}$ is multiplied,
not necessarily explicitly, by matrices or by a vector. 

Therefore, the programming interface of UltimateKalman expects input
covariance matrices to be represented as objects belonging to a type
with a method \texttt{weigh} that multiplies the factor $W$ by a
matrix $A$ or a vector $v$. In the MATLAB and Java implementations,
this type is called \texttt{CovarianceMatrix}. The constructors of
these classes accept many representations of a covariance matrix:
\begin{itemize}
\item An explicit covariance matrix $C$; the constructor computes an upper
triangular Cholesky factor $U$ of $C=U^{T}U$ and implements \texttt{X=C.weigh(A)}
by solving $UX=A$.
\item An inverse factor $W$ such that $W^{T}W=C^{-1}$; this factor is
stored and multiplied by the argument of \texttt{weigh}.
\item An inverse covariance matrix $C^{-1}$; the constructor computes its
Cholesky factorization and stores the lower-triangular factor as $W$.
\item A diagonal covariance matrix represented by a vector $w$ such that
$W=\text{diag}(w)$ (the elements of $w$ are inverses of standard
deviations).
\item A few other, less important, variants.
\end{itemize}
In the MATLAB implementation, the way that the argument to the constructor
represents $C$ is defined by a single-character argument (with values
\texttt{C}, \texttt{W}, \texttt{I}, and \texttt{w}, respectively).
In the Java implementation, \texttt{CovarianceMatrix} is an interface
with two implementing classes, \texttt{DiagonalCovarianceMatrix} and
\texttt{RealCovarianceMatrix}; the way that the numeric argument represents
$C$ is specified using \texttt{enum} constants defined in the implementation
classes:
\begin{lstlisting}
RealCovarianceMatrix.Representation.COVARIANCE_MATRIX
RealCovarianceMatrix.Representation.FACTOR
RealCovarianceMatrix.Representation.INVERSE_FACTOR
DiagonalCovarianceMatrix.Representation.COVARIANCE_MATRIX
DiagonalCovarianceMatrix.Representation.DIAGONAL_VARIANCES
DiagonalCovarianceMatrix.Representation.DIAGONAL_STANDARD_DEVIATIONS
DiagonalCovarianceMatrix.Representation.DIAGONAL_INVERSE_STANDARD_DEVIATIONS
\end{lstlisting}
The \texttt{RealCovarianceMatrix} class has a single constructor that
takes a \texttt{RealMatrix} and a representation constant. The \texttt{DiagonalCovarianceMatrix}
class several constructors that take either a \texttt{RealVector},
an array of \texttt{double} values, or a single \texttt{double} and
a dimension (the covariance matrix is then a scaled identity); all
also take as a second argument a representation constant.

Covariance input matrices are passed to the C implementation in a
similar manner, but without a class; each input covariance matrix
is represented using two arguments, a matrix and a single character
(\texttt{C}, \texttt{W}, \texttt{I}, or \texttt{w}) that defines how
the given matrix is related to $C$. 

UltimateKalman always returns the covariance matrix of $\estimate u_{i}$
as an upper-triangular inverse factor $W$. The MATLAB and Java
implementations return covariance matrices as objects of the \texttt{CovarianceMatrix}
type (always with an inverse-factor representation); the C implementation
simply returns the inverse factor as a matrix.

\section{\label{sec:matlab}The MATLAB programming interface}

The MATLAB implementation resides in the \texttt{matlab} directory
of the distribution archive. Add it to your MATLAB search path using
MATLAB \texttt{addpath} command. 

The MATLAB implementation is object oriented and is implemented as
a handle (reference) class. The constructor takes no arguments.

\begin{lstlisting}
kalman = UltimateKalman()
\end{lstlisting}

The (overloaded) methods that advance the filter through a sequence
of steps are \texttt{evolve} and \texttt{observe}. Each of them must
be called exactly once at each step, in this order. The \texttt{evolve}
method declares the dimension of the state of the next step and provides
all the known quantities of the evolution equation, 

\begin{lstlisting}
kalman.evolve(n_i, H_i, F_i, c_i, K_i)
\end{lstlisting}
where \texttt{n\_i} is an integer, the dimension of the state, \texttt{H\_i}
and \texttt{F\_i} are matrices, \texttt{c\_i} is a vector, and \texttt{K\_i}
is a \texttt{CovarianceMatrix} object. The number of rows in\texttt{
H\_i}, \texttt{F\_i}, and \texttt{c\_i} must be the same and must
be equal to the order of \texttt{K\_i}; this is the number $\ell_{i}$
of scalar evolution equations. The number of columns in \texttt{H\_i}
must be \texttt{n\_i} and the number of columns in \texttt{F\_i} must
be equal to the dimension of the previous step. A simplified overloaded
version defines \texttt{H\_i} internally as an $n_{i}$-by-$n_{i-1}$
identity matrix, possibly padded with zero columns

\begin{lstlisting}
kalman.evolve(n_i, F_i, c_i, K_i)
\end{lstlisting}
If $n_{i}>\ell_{i}$, this overloaded version adds the new parameters
to the end of the state vector.

If $n_{i}<\ell_{i}$, the first version must be used; this forces
the user to specify how parameters in $u_{i-1}$ are mapped to the
parameters in $u_{i}$. The \texttt{evolve} method must be called
even in the first step; this design decision was taken mostly to keep
the implementation of all the steps in client code uniform. In the
first step, there is no evolution equation, so the user can pass empty
matrices to the method, or call another simplified overloaded version:

\begin{lstlisting}
kalman.evolve(n_i)
\end{lstlisting}

The \texttt{observe} method comes in two overloaded versions. One
of them must be called to complete the definition of a step. The first
version describes the observation equation and the second tells UltimateKalman
that there are no observations of this step.

\begin{lstlisting}
kalman.observe(G_i, o_i, C_i)
kalman.observe()
\end{lstlisting}

Steps are named using zero-based integer indices; the first step that
is defined is step~$i=0$, the next is step~$1$, and so on. The
\texttt{estimate} methods return the estimate of the state at step
\texttt{i} and optionally the covariance matrix of that estimate,
or the estimate and covariance of the latest step that is still in
memory (normally the last step that was observed):

\begin{lstlisting}
[estimate, covariance] = kalman.estimate(i)
[estimate, covariance] = kalman.estimate()
\end{lstlisting}
If a step is not observable, \texttt{estimate} returns a vector of
$n_{i}$ \texttt{NaN}s (not-a-number, an IEEE-754 floating point representation
of an unknown quantity). 

The \texttt{forget} methods delete from memory the representation
of all the steps up to and including $i$, or all the steps except
for the latest one that is still in memory.

\begin{lstlisting}
kalman.forget(i)
kalman.forget()
\end{lstlisting}
The \texttt{rollback} methods return the filter to its state just
after the invocation of \texttt{evolve} in step \texttt{i}, or just
after the invocation of \texttt{evolve} in the latest step still in
memory.

\begin{lstlisting}
kalman.rollback(i)
kalman.rollback()
\end{lstlisting}
The methods \texttt{earliest} and \texttt{latest} are queries that
take no arguments and return the indices of the earliest and latest
steps that are still in memory.

The \texttt{smooth} method, which also takes no arguments, computes
the smoothed estimates of all the states still in memory, along with
their covariance matrices. After this method is called, \texttt{estimate}
returns the smoothed estimates. A single step can be smoothed many
times; each smoothed estimate will use the information from all past
steps and the information from future steps that are in memory when
\texttt{smooth} is called.

To use the C or Java implementations from within MATLAB, create
an object of one of the adapter classes, not an object of the \texttt{UltimateKalman}
class:
\begin{lstlisting}
kalman = UltimateKalmanNative()
kalman = UltimateKalmanJava()
\end{lstlisting}
To use the C implementation, you will first need to compile the C
code into a MATLAB-callable dynamically-linked library that MATLAB
uses through an interface called the \texttt{mex} interface. To perform
this step, run the \texttt{compile.m} script in the \texttt{matlab}
directory. To use the Java implementation, build the Java library
using the instructions in the next section, and add both the resulting
library (a \texttt{jar} file) and the library containing the Apache
Commons Math library to the MATLAB search path using MATLAB \texttt{addpath}
command. The script \texttt{replication.m} adds these libraries to
the path and can serve as an example.

\section{\label{sec:java}The Java programming interface}

The programming interface to the Java implementation is nearly identical.
The implementing class is \texttt{sivantoledo\LyXZeroWidthSpace .kalman\LyXZeroWidthSpace .UltimateKalman}.
It also uses overloaded methods to express default values. It differs
from the MATLAB interface only in that the \texttt{estimate} methods
return only one value, the state estimate. To obtain the matching
covariance matrix, client code must call a separate method, \texttt{covariance}.

The Java implementation resides under the \texttt{src} directory
of the distribution archive. To use it, you first need to compile
the source code and to assemble the compiled code into a library (a
\texttt{jar} file). The scripts \texttt{build.bat} and \texttt{build.sh},
both in the top directory of the distribution archive, perform these
steps under Windows (\texttt{build.bat}) and Linux and MacOS (\texttt{build.sh}).
To run the scripts, your computer must have a Java development kit
(JDK) installed. We used successfully releases of OpenJDK on both
Windows and Linux. The library is build so that it can be used with
any version of Java starting with version~8 (sometimes also referred
to as 1.8).

The build scripts also compile and run a test program, \texttt{Rotation.java}.
It performs the same computations and outputs the same values as the
MATLAB example function \texttt{rotation.m} when executed with arguments
\texttt{rotation(UltimateKalman,5,2)}. This program also serves as
an example that shows how to write Java code that calls UltimateKalman.

\section{\label{sec:c-lang}The C programming interface}

In the C interface, the filter is represented by a pointer to a structure
of the \texttt{kalman\_t} type; to client code, this structure is
opaque (there is no need to directly access its fields). The filter
is constructed by a call to \texttt{kalman\_create}, which returns
a pointer to \texttt{kalman\_t}. 

In general, the memory management principle of the interface (and
the internal implementation) is that client code is responsible for
freeing memory that was allocated by a call to any function whose
name includes the word \texttt{create}. Therefore, when client code
no longer needs a filter, it must call \texttt{kalman\_free} and pass
the pointer as an argument. 

The functionality of the filter is exposed through functions that
correspond to methods in the MATLAB and Java implementations. These
functions expect a pointer to \texttt{kalman\_t} as their first argument.
The functions are not overloaded because C does not support overloading.
Missing matrices and vectors (e.g., to \texttt{evolve} and \texttt{observe})
are represented by a \texttt{NULL} pointer and default step numbers
(to \texttt{forget} , \texttt{estimate}, and so on) by $-1$. Here
is the declaration of some of the functions. 
\begin{lstlisting}
kalman_t*        kalman_create    ();
void             kalman_free      (kalman_t* kalman);
void             kalman_observe   (kalman_t* kalman,                                                               
                                   kalman_matrix_t* G_i, kalman_matrix_t* o_i,
                                   kalman_matrix_t* C_i, char C_i_type);
int64_t          kalman_earliest  (kalman_t* kalman);
void             kalman_smooth    (kalman_t* kalman);
kalman_matrix_t* kalman_estimate  (kalman_t* kalman, int64_t i);
kalman_matrix_t* kalman_covariance(kalman_t* kalman, int64_t i);
void             kalman_forget    (kalman_t* kalman, int64_t i);
...
\end{lstlisting}
Note that input covariance matrices are represented by a \texttt{kalman\_matrix\_t}
and a representation code (a single character). The output of \texttt{kalman\_covariance}
is a matrix $W$ such that $(W^{T}W)^{-1}$ is the covariance matrix
of the output of \texttt{kalman\_estimate} on the same step.

A small set of helper functions allows client code to construct input
matrices in the required format, to set their elements, and to read
and use matrices returned by UltimateKalman. Here are the declarations
of some of them.
\begin{lstlisting}
kalman_matrix_t* matrix_create(int32_t rows, int32_t cols);
void             matrix_free(kalman_matrix_t* A);

void   matrix_set(kalman_matrix_t* A, int32_t i, int32_t j, double v);
double matrix_get(kalman_matrix_t* A, int32_t i, int32_t j);

int32_t matrix_rows(kalman_matrix_t* A);
int32_t matrix_cols(kalman_matrix_t* A);
...
\end{lstlisting}

Client code is responsible for freeing matrices returned by \texttt{kalman\_estimate}
and \texttt{kal\-man\_\-covariance} by calling \texttt{matrix\_free}
when they are no longer needed.

\subsection*{Building and Running the Code Outside Matlab}

To help you integrate UltimateKalman into your own native code in
C or C++, the software includes two C programs that call UltimateKalman
and Windows and Linux scripts that builds executable versions of the
two programs. The scripts compile one of the programs along with UltimateKalman,
link it with BLAS and LAPACK libraries, and run it. You should be
able to use these scripts as examples of how to compile and link UltimateKalman.
In particular, the C implementation of UltimateKalman consists of
a single C source file, so it is not necessary to build it into a
library; the source code or a single object file can be used directly
in your C or C++ programs.

These C programs, the C implementation of UltimateKalman, and the
build scripts are all under the \texttt{native} directory of the distribution
archive.

The two sample programs are \texttt{rotation.c}, a C implementation
of one of the example MATLAB programs, and \texttt{blastest.c}, a
small program intended only to test the interface to the BLAS and
LAPACK libraries. The program \texttt{rotation.c} performs the same
computation that the expression \texttt{rotation(UltimateKalman,5,2)}
performs in MATLAB and should output the same numerical results,
just like \texttt{Rotation.java} in Java.

The Windows script, \texttt{build.bat}, uses the Microsoft C command-line
compiler (\texttt{cl}) that comes with Microsoft's Visual Studio Community
2022~\cite{VSCommunity}, a free integrated development environment,
and the BLAS and LAPACK libraries that are part of Intel's free
oneAPI Math Kernel Library (MKL)~\cite{MKL}. The script takes one
argument, the name of the program that you want to build and run,
\texttt{blastest} or \texttt{rotation}. Here what you should expect
to see on the console when you build and run \texttt{blastest} (ellipsis
stand for deleted output that is not particularly interesting; the
output of the \texttt{blastest} program is shown in bold):
\begin{lstlisting}
C:\Users\stoledo\github\ultimate-kalman\native>build.bat blastest
...
**********************************************************************
** Visual Studio 2022 Developer Command Prompt v17.9.2
** Copyright (c) 2022 Microsoft Corporation
**********************************************************************
[vcvarsall.bat] Environment initialized for: 'x64'
:: initializing oneAPI environment...
   Initializing Visual Studio command-line environment...
   Visual Studio version 17.9.2 environment configured.
   "C:\Program Files\Microsoft Visual Studio\2022\Community\"
:  compiler -- latest
:  mkl -- latest
:  tbb -- latest
:: oneAPI environment initialized ::
generating test program blastest.exe
ultimatekalman.c
ultimatekalman.c(39): warning C4005: 'blas_int_t': macro redefinition
ultimatekalman.c(28): note: see previous definition of 'blas_int_t'
blastest.c
Generating Code...
generated test program
~BLAS test starting~
~A = matrix_print 2 3~
~1 2 3~
~4 5 6~
~B = matrix_print 3 2~
~7 8~
~9 10~
~12 13~
~C = matrix_print 2 2~
~125 137~
~293 323~
~Result should be:~
~  125  137~
~  293  323~
~BLAS test done~
done running test
build script done
\end{lstlisting}

When you run the Windows executable program yourself, make sure that
the directory where the MKL dynamically-linked libraries (\texttt{dll}
files) are stored is on your \texttt{path}. The same is true, of course,
for your own programs that use UltimateKalman. The installation of
MKL does not modify the path to include this library, but it does
provide a script that modifies the search path appropriately. In the
version of MKL that I used, this script is \texttt{c:\textbackslash Program
Files (x86)\textbackslash Intel\textbackslash oneAPI\textbackslash setvars.bat}. 

Under Linux, the corresponding script is \texttt{build.sh}. It also
takes a single argument, \texttt{blastest} or \texttt{rotation}. The
script provides with the software uses the BLAS and LAPACK libraries
that are installed in the Ubuntu Linux distribution using the commands
\begin{lstlisting}
sudo apt install libblas-dev
sudo apt install liblapack-dev
\end{lstlisting}
These software packages install the libraries in a directory that
is already on the search path, so no special configuration of the
search path is required. These particular implementations of the BLAS
and LAPACK are not high-performance implementations, but since UltimateKalman
typically handles fairly small matrices, a high-performance library
like Intel's MKL may not provide performance benefits. However, MKL
is also available for Linux and you can certainly use it.

To successfully link UltimateKalman with other implementations of
the BLAS and LAPACK, you may need to set some preprocessor variables
that control how UltimateKalman calls these libraries. Most of the
variations are due to the fact that these libraries were originally
implemented in Fortran. There are a few other preprocessor variables
that control the behavior of UltimateKalman. All of these are listed
and explained in Table~\ref{tab:C-pre-variables}.

\begin{table}
\begin{centering}
\begin{tabular}{ll>{\raggedright}p{0.4\textwidth}}
variable name & requires a value? & explanation\tabularnewline
\hline 
\texttt{BUILD\_MKL} & no & specifies that MKL is used; sets all the other BLAS and LAPACK
variables (so you do not need to)\tabularnewline
\texttt{BUILD\_BLAS\_INT} & yes & C data type that specifies a row or column index or a dimension of
a matrix; usually either \texttt{int32\_t} or \texttt{int64\_t}\tabularnewline
\texttt{HAS\_BLAS\_H} & no & specifies that the \texttt{blas.h} header is available\tabularnewline
\texttt{HAS\_LAPACK\_H} & no & specifies that the \texttt{lapack.h} header is available\tabularnewline
\texttt{BUILD\_BLAS\_UNDERSCORE} & no & add an underscore to names of BLAS and LAPACK functions\tabularnewline
\texttt{BUILD\_BLAS\_STRLEN\_END} & no & add string-length arguments to calls to the BLAS and LAPACK\tabularnewline
\texttt{BUILD\_DEBUG\_PRINTOUTS} & no & generates run-time debug printouts\tabularnewline
\texttt{NDEBUG} & no & suppresses run-time assertion checking\tabularnewline
\texttt{BUILD\_WIN32\_GETTIMEOFDAY} & no & required under Windows; do not use under Linux or MacOS\tabularnewline
\end{tabular}
\par\end{centering}
\caption{\label{tab:C-pre-variables}Preprocessor variables that control how
UltimateKalman calls the BLAS and LAPACK, as well as several other
aspects of its behavior.}
\end{table}


\section{\label{sec:testing-and-examples}Testing and Code Examples }

The software distribution of UltimateKalman includes five MATLAB
functions that demonstrate how to use UltimateKalman:
\begin{itemize}
\item \texttt{rotation.m}, modeling a rotating point in the plane (this
example is also implemented in C and Java, as described above).
\item \texttt{constant.m}, modeling an fixed scalar or a scalar that increases
linearly with time, and with observation covariance matrices that
are identical in all steps except perhaps for one exceptional step.
\item \texttt{add\_remove.m}, demonstrating how to use $H_{i}$ to add or
remove parameters from a dynamic system.
\item \texttt{projectile.m}, implementing the model and filter of a projectile
described by Humpherys et al.~\cite{AFreshLook2012}.
\item \texttt{clock\_offsets.m}, implementing clock-offset estimation in
a distributed sensor system. This example demonstrates how to handle
parameters that appear only in one step.
\end{itemize}
The mathematical details of these models are described in the article
that describes UltimateKalman.

The script \texttt{replication.m} runs all of these examples and optionally
generates the graphs shown in the article. The script can run not
only the MATLAB implementation, but also the C and Java implementations.

\section{\label{sec:performance-testing}Support for Performance Testing}

All the implementations include a method, called \texttt{perftest},
designed for testing the performance of the filter. This method accepts
as arguments all the matrices and vectors that are part of the evolution
and observation equations, a step count, and an integer $d$ that
tells the method how often to take a wall-clock timestamp. The method
assumes that the filter has not been used yet and executes the filter
for the given number of steps. In each step, the state is evolved
and observed, the state estimate is requested, and the previous step
(if there was one) is forgotten. The same fixed matrices and vectors
are used in all steps.

This method allows us to measure the performance of all the implementations
without the overheads associated with calling C or Java from MATLAB.
That is, the C functions are called in a loop from C, the Java
methods are called from Java, and the MATLAB methods from MATLAB.

The method takes a timestamp every $d$ steps and returns a vector
with the average wall-clock running time per step in each nonoverlapping
group of $d$ steps.

\section{\label{sec:step-data-structure}Data Structures for the Step Sequence}

The information in this section helps to understand the implementations,
but is essentially irrelevant to users of UltimateKalman.

The Java implementation uses an \texttt{ArrayList} data structure
to represent the sequence of steps that have not been forgotten or
rolled back, along with an integer that specifies the step number
of the first step in the \texttt{ArrayList}. The data structure allows
UltimateKalman to add steps, to trim the sequence from both sides,
and to access a particular step, all in constant time or amortized
constant time.

The C implementation uses a specialized data structure with similar
capabilities. This data structure, called in the code \texttt{farray\_t},
is part of UltimateKalman. The sequence is stored in an array. When
necessary, the size of the array is doubled. The active part of the
array is not necessarily in the beginning, if steps have been forgotten.
When a step is added and there is no room at the end of the physical
array, then either the array is reallocated at double its current
size, or the active part is shifted to the beginning. This allows
the data structure to support appending, trimming from both sides,
and direct access to a step with a given index, again in constant
or amortized constant time.

The Java implementation stores the steps in a cell array. The implementation
is simple, but not as efficient as the data structure that is used
by the C version.

\bibliographystyle{plainurl}
\bibliography{kalman}

\end{document}
